\documentclass[10pt,landscape]{article}
\usepackage[utf8]{inputenc}
\usepackage{multicol}
\usepackage{calc}
\usepackage{ifthen}
\usepackage[landscape]{geometry}
\usepackage{hyperref}
\usepackage{pifont}

% To make this come out properly in landscape mode, do one of the following
% 1.
%  pdflatex latexsheet.tex
%
% 2.
%  latex latexsheet.tex
%  dvips -P pdf  -t landscape latexsheet.dvi
%  ps2pdf latexsheet.ps

% The basic layout is taken from a Latex Cheat Sheat .tex file from 
% Winston Chang http://wch.github.io/latexsheet/

% This sets page margins to .5 inch if using letter paper, and to 1cm
% if using A4 paper. (This probably isn't strictly necessary.)
% If using another size paper, use default 1cm margins.
\ifthenelse{\lengthtest { \paperwidth = 11in}}
	{ \geometry{top=.5in,left=.5in,right=.5in,bottom=.5in} }
	{\ifthenelse{ \lengthtest{ \paperwidth = 297mm}}
		{\geometry{top=1cm,left=1cm,right=1cm,bottom=1cm} }
		{\geometry{top=1cm,left=1cm,right=1cm,bottom=1cm} }
	}

% Turn off header and footer
\pagestyle{empty}
 

% Redefine section commands to use less space
\makeatletter
\renewcommand{\section}{\@startsection{section}{1}{0mm}%
                                {-1ex plus -.5ex minus -.2ex}%
                                {0.5ex plus .2ex}%x
                                {\normalfont\large\bfseries}}
\renewcommand{\subsection}{\@startsection{subsection}{2}{0mm}%
                                {-1explus -.5ex minus -.2ex}%
                                {0.5ex plus .2ex}%
                                {\normalfont\normalsize\bfseries}}
\renewcommand{\subsubsection}{\@startsection{subsubsection}{3}{0mm}%
                                {-1ex plus -.5ex minus -.2ex}%
                                {1ex plus .2ex}%
                                {\normalfont\small\bfseries}}
\makeatother

% Don't print section numbers
\setcounter{secnumdepth}{0}


\setlength{\parindent}{0pt}
\setlength{\parskip}{0pt plus 0.5ex}


% -----------------------------------------------------------------------

\begin{document}

\raggedright
\footnotesize
\begin{multicols}{3}


% multicol parameters
% These lengths are set only within the two main columns
%\setlength{\columnseprule}{0.25pt}
\setlength{\premulticols}{1pt}
\setlength{\postmulticols}{1pt}
\setlength{\multicolsep}{1pt}
\setlength{\columnsep}{2pt}

\begin{center}
     \Large{\textbf{FORTRAN Reference Card}} \\
\end{center}

\section{Program Structure}
\newlength{\MyLen}
\settowidth{\MyLen}{\texttt{.begin.equation..place.}}
\begin{tabular}{@{}p{\the\MyLen}%
                @{}p{\linewidth-\the\MyLen}@{}}
\verb!PROGRAM! \textit{name}        &  Begin program \textit{name}    \\
\verb!END PROGRAMM! \textit{name}   &                     \\
\verb!SUBROUTINE! \textit{name}     &  Begin subroutine \textit{name} \\
\verb!END SUBROUTINE! \textit{name} &  \\
\verb!MODULE! \textit{name}         & Begin module \textit{name} \\
\verb|END MODULE| \textit{name}                    
\end{tabular}

\section{Fortran Preprocessors}
\begin{tabular}{@{}ll@{}}
\verb!IMPLICIT NONE!       &  Avoid using predefined data types 
\end{tabular}

\section{Intrinsic Data Types}
\begin{tabular}{@{}ll@{}}
\verb!INTEGER!       &  Number without fractional part\\
\verb!REAL!          &  Floating-point format         \\
\verb!COMPLEX!       &  \verb|(a,b)| Complex number\\
\verb!CHARACTER!     &  String of characters enclosed in \verb!'! or \verb!"!\\
\verb!LOGICAL!       &  \verb|.true.| or \verb|.false.|
\end{tabular}
\section{Derived Data Types}
\begin{tabular}{@{}ll@{}}
\verb|TYPE| \textit{type1}   & Define a new structure \\
\verb|END TYPE|             & \\
\verb|TYPE, EXTENDS(type1) ::| \textit{type2} & Extending an existing type\\
\verb|TYPE(type1) :: name | & Create type \\
\verb|type1%component|      & Access component of type
\end{tabular} \\
\textit{Example:} \\
\begin{verbatim}
type Books
  character(len = 50) :: title
  integer :: book_id
end type Books
type(Books) :: book1 
book1%title = "Night"
book1%book_id = 1
\end{verbatim}

\subsection{Additional Attributes}
\begin{tabular}{@{}ll@{}}
\verb!KIND! \textit{= val} &  Define presicion of a real \\
\hspace{2mm} \textit{val} = 4 (32bit), 8(64bit) & GNU Fortran compiler \\
\verb!PARAMETER! & Value is set to be constant \\
\verb!DIMENSION! & Assign dimension to an object \\
\verb!POINTER!   & Object will be pointer to content \\
\verb|TARGET|    & Object is target for a pointer \\
\verb!ALLOCATABLE! & Object can be allocated \\
\verb!PRIVATE!   & Only access object in module \\
\verb!PUBLIC!    & Not privat\\
\end{tabular} \\

\textit{Example:} \verb! INTEGER, PARAMETER :: x, value! \\
\hspace{14mm} \verb!CHARACTER(len=20) :: name!
Fortran is \textit{case insensitive}. \\

\section{Arrays}


\section{Miscellaneous}
\begin{tabular}{@{}ll@{}}
\verb@!@ 		& Comment (older versions: \verb!C!)	\\
\verb!&! 		& Continue statement in new line    
\end{tabular}
\\
\textbf{Statement labels} are numbers without meaning but
 they can be used to refer to a statement.\\
\textit{Example:} \verb! 100 output = x + y ! \\ % This is old fortran

\section{Flow Control}
\begin{tabular}{@{}ll@{}}
\verb!DO! \textit{name}              &  While loop       \\
\hspace{2mm}\verb!IF (! \textit{logical\_expr} \verb!) EXIT!  &  Exit condition \\
\verb!END DO!    					 & 					 \\
\verb!DO index = istart, iend, incr! & Iterative do loop \\
\hspace{2mm} \verb!Statements!       &					 \\
\verb!END DO! 						 &                   
\end{tabular} \\

Loops and branching statements can have names. \\
\textit{Example:} \verb!loopname: DO [..] END DO loopname! \\

\settowidth{\MyLen}{\texttt{.begin.equation..place.}}
\begin{tabular}{@{}p{\the\MyLen}%
                @{}p{\linewidth-\the\MyLen}@{}}
\verb!STOP!\textit{'optional string'}& Terminate program              \\
\verb!ERROR STOP! \textit{'error msg'} & Informs system that program failed after terminating \ding{95} \\
\end{tabular} \\
The \verb!STOP! statement is more or less reduntant.


\section{Operators}
Operations beginning with highest in hierarchy.\\
\begin{tabular}{@{}l@{\hspace{1em}}l@{\hspace{2em}}l@{\hspace{1em}}l@{}}
Exponentiation       & \verb!**!     & &    \\
Multiplication       & \verb!*!             & Division             & \verb!/ !           \\  
Addition             & \verb!+!             & Subtraction          & \verb!-!           \\
\end{tabular}
\begin{tabular}{@{}ll@{}}
	\verb|=>|   &   Assign target/object to pointer
\end{tabular}


\section{Important Functions}
%describe them properly
\begin{tabular}{@{}ll@{}}
\verb!date_and_time!       & Get the date and time			 \\
\verb!random_seed(size=k)! & Get a random number of size k   \\                   
\end{tabular} \\
External functions are called with \verb!CALL!. \\
\subsection{Math Functions}
\begin{tabular}{@{}ll@{\hspace{1.3em}}l@{\hspace{0.5em}}}
\verb!INT(x)!         	& Integer part of x 			& \verb!INT(2.95)! $\rightarrow$ 2  \\
\verb!NINT(x)!        	& Round x 					 	& \verb!NINT(2.95)! $\rightarrow$ 3  \\ 
\verb!CEILING(x)!     	& Nearest integer above x		& \verb!CEILING(2.95)! $\rightarrow$ 3	 \\
\verb!FLOOR(x)!       	& Nearest integer below x 		& \verb!FLOOR(2.95)! $\rightarrow$ 2	 \\
\verb!REAL(i)!        	& Convert integer to real       &  \\
\hline          
\end{tabular} \\
\begin{tabular}{@{}ll@{}}
\verb!SQRT(x)!			& Square root of x for x $\geq 0$	\\
\verb!ABS(x)!			& Absolute value of x 				\\
\verb!SIN(x), SIND(x)!	& Sine of x (in radians, degrees)	\\
\verb!COS(x), COSD(x)!  & Cosine of x (radians, degree)		\\
\verb!TAN(x), TAND(x)! 	& Tangent of x (radians, degree)	\\
\verb!EXP(x)!			& e to the xth Power				\\
\verb!LOG(x), LOG10(x)!	& Natural logarithm, Base 10-logarithm \\
\verb!MOD(a,b)!			& Modulo function					\\
\verb!MAX(a,b), MIN(a,b)! & Pickes larger/smaller of a and b \\
\verb!NORM2(array)!     & Calculate Euclidean norm ($L_2$ norm)\ding{95} \\
\verb!ERF(x), ERFC(X)! 	& (Complementary) Error function  \ding{95}	\\
\end{tabular} \\


\section{Input/Output}
% possibly good idea to seperate in file and standard I/O
% And try to say when you would use what.
\begin{tabular}{@{}ll@{}}
\verb!WRITE (*,*)! & Print to standard output stream \\
\verb!READ (*,*)!  & Read from standard input stream \\
\verb!PRINT *,!    & Print to standard output stream \\
\end{tabular}

\subsection{Formating I/O}

\section{Compilation gfortran}
Using gfortran on a UNIX-like system. \\ 
\verb!gfortran myprogram.f -o myprogram.out! \\
\textbf{File extensions} (recommendation is first one): \\
\verb!file.f90! free-form source, no preprocessing\\
\verb!file.F90! free-form source, preprocessing \\
\verb!file.f! fixed-form source, no preprocessing\\
\verb!file.F! fixed-form source, preprocessing \\
\textbf{Several files:} \\
First compile subfiles \verb!gfortran -c module.f90!
Then all \verb!gfortran main.f90 module.o -o main.o! \\
\textbf{Other Options:} \\
\settowidth{\MyLen}{\texttt{.begin.equation..place...}}
\begin{tabular}{@{}p{\the\MyLen}%
			@{}p{\linewidth-\the\MyLen}@{}}
\verb!-std=f95! & Set standard for compiler           \\
                & (\verb!f2003, f2008, gnu=default, legacy!) \\
\verb!-Wextra -Wall -pedantic! & Recommended warnings\\
\verb|-c|       & Necessary if only module or subroutine compilation \\
\end{tabular}

\section{Comments}
Variable types are indicated by the used character: \\
x = real; i= int; a,b= int/real \\
Functions from standard after 95 are marked with a \ding{95} .\\
%\textbf{Abbreviations:} int=integer; I/O= Input/Output\\
% In case they are not clear.

%---------------------------------------------------------------------------

\rule{0.3\linewidth}{0.25pt}
\scriptsize

\the\year ~ Tracy Kiszler; No guarantee for anything :)\\

\href{https://github.com/TracyMcBean/Fortran\_cheatsheet}{https://github.com/TracyMcBean/Fortran\_cheatsheet}


\end{multicols}
\end{document}
